\section{Structured approach}
\todo{write out the list below}
\begin{itemize}
	\item{Figure out the required knowledge and technologies}
	\item{Draft the (measurable/verifiable) measure for success}
	\item{Related work: which similar problems are already (partly) solved?}
	\item{Design a concept}
	\item{Evaluate the concept theoretically}
	\item{Exemplary show the realization of the concept to confirm the theoretical evaluation}
\end{itemize}

To archive the aforementioned goal a structured approach was chosen in such a way to be comprehensible and verifiable. This chapter will give an outline of that approach.

In the first step research was done to identify the required knowledge and technologies. This especially includes a current state of the art regarding the research on topics like robotic exploration, swarm coordination, probabilistic robotics, measurement filtering and more. Also, the robots used in the given setup were closely examined for their capabilities and limits, as well as the way to control them and extract data. A special focus was set on swarm capabilites during that process.

In the next step the goal of this work was defined. Within the scope of that a reasearch question and hypothesis was definied along with the measure for the hypothesis to be true or false. This took into account existing works in the field and the questions already answered or asked by them.

These existing works were also regarded for steps in this work to re-use and build upon. Out of that a concept was developed to address the previously defined goal. That included comparing different approaches for aspects like movement, measurement and processing of combined swarm-data.

Firstly, this concept then got evaluated theretically in comparing it to alternatives \todo{which?}.

Next, the concept also was evaluated in an exemplary implementation and conducted experiments. The results from that were compared to the theory and finally assessed in regards of the initial goal and it's measure for success.
