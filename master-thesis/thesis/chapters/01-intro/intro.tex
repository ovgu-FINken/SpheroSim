\todo{write out the list below}
\begin{itemize}
	\item{Importance of robotics}
	\item{Importance of swarm-robotics}
	\item{We can't measure \emph{true/exact} values --> probabilistic approach}
	\item{Combination of "swarm" and "probabilistic" to find good solutions despite of the problems}
\end{itemize}

Robots and automation are topics that more then ever take place in the average everyday life. Not only are robots well-established in industrial surroundings but also personal appliances are getting common, like autonomous vacuum-robtos or self-driving cars. It's obvious that robots can assist humans on their tasks or even replace the need for humans, sometimes even surpassing human performance or accuracy.

In the recent years robots became so common that due tue improvements in their manufacturing and pricing it is now possible to get elaborated robots as toys for personal usage. This again opened up new posibilies in regards of swarm-robotics.

The field of swarm-intelligence traditionally was inspired by nature itself, seeing large numbers of individual animals like fish or birds behave in such an manner that the combined actions of all the single individuals result in an action none of the individuals could have accomplished on it's own. Many of the theories on swarm-behavior are well known for years. Now with the emerging possibilies introduced by cheap, simple robots these theories can be put to test by experimenting in real-world setups with adjustable individuals. The first results of this research can lately be observed in the form of \todo{find some popular, demonstrative examples}.

One persisting problem in the field of real-world robotics is the inherent inaccuracy of the robots actions, it's sensorics and even their surrounding. The models used to define the behavior of the robots often times assumed exact values and for most simple scenarios the observed real-world results were close enough or the deviation could be explained and accounted for. But with increasing complexity of the use-cases the robots are applied in this inaccuracy becomes a bigger problem.

As a solution for this, the models defining the behaviour of the robots tried not to calculate with exact values, as these were known to be unrealistic. Instead a probabilistic apporach was chosen. In that, the varying confidence in the percieved values is accounted for and all the possible results of the possible values are computed each with a probability. This is results not in a single value which might or might not be fitting, but instead in a distribution of possible values and their respective probability to be true. Future iterations of usage or perception of that value then are used to improve and adjust that distribution of probability.

Any usage of these probabilistic data is subject to uncertainty. As a result it is a key factor to improve the probabilistic data to their maximum possible quality. The iterative approach of constantly improving the data indicates a correlation between the number of samples used and the overall quality of the estimate. Here is where Swarm-robotics enters the stage \todo{wording}. One of swarm-robotic's key features is a high number of individuals. These individuals are able to cross-check each other's perception and when feeding into a shared estimation can easily outperform a greater number of autarkic individuals, each only acting on it's own.